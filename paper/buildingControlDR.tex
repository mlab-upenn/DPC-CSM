\section{Application -- building control and demand response}

\begin{itemize}
	\item intro to building control and demand response
	\item need for model predictive control
	\begin{itemize}
		\item energy efficiency, energy flexibility \(->\) energy savings, cost savings
	\end{itemize}
	\item so we need models, why is traditional way of modeling hard 

	\begin{itemize}
		\item model capture using historical data 

		\item change in material properties 

		\item model heterogeneity
	\end{itemize}
	\item Practical challenges
	\begin{itemize}
		\item quality of historical data, need for new experiments, sensor failure 

		\item computational complexity of control/optimization algorithms, real-time control 

		\item performance guarantees and robustness 

		\item model adaptability
		\item indicator for deterioration, when to update, use statistics of error in prediction
	\end{itemize}
	\item A concrete example that describes the modeling and control problem
	\begin{itemize}
		\item description of building: different types of buildings like RTU, central heating/cooling, impact of thermal inertia

		\item goals for modeling — types of models to be identified 

		\item goals for control — cost minimization, energy minimization, thermal comfort bounds 
	\end{itemize}
\end{itemize}